\chapter{Introduction}
\label{ChapterI}

The aftermath of the Great Financial Crisis of 2008-2009 has shown that while most mainstream models in economics treated financial institutions as mere value-neutral plumbing - albeit plumbing that grew at rates much faster than the house in the words of Dr. Mark Blythe \todo{youtube talk citation}; it is ignored at their own peril.  This brings us to the research - specifically in geography - with regards to institutional investors. There hasn't been a lot of new literature in geography about the locational preferences of institutional investors during the 21st century. Institutional investors are individuals (corporate or natural beings) that exercise investment discretion over large sums of money\footnote{The Securities and Exchange Commission placed the reporting threshold for institutional investors at 100 million USD}\citep{SEC2013}.  The idea behind pooling funds to obtain larger risk-adjusted returns stems from the old advice against putting all of one's eggs in one basket, and the larger pool of funds allows for putting more eggs in different baskets.  This age-old advice was formalized by \cite{Markowitz1952} in his seminal paper on Modern Portfolio Theory.  


As such, with a few notable exceptions such as \cite{Graves2003,gongthe2012} and \cite{GreenOLef2014}, most of the Geography based literature for institutional investment dates to the 1980s and 1990s.  At the same time, the 2000s and the 2010s has seen some economics and finance articles that investigated the role of space and place with regards to investing, most of these papers research question is more interested in drawing a competitive advantage rather than survey the location of investors.  Part of this can be explained by the culture turn in Geography and its shifting the meta narrative of research away from quantitative surveys to more personalized explorations of space and place as well as the reduced perceived importance of classical location theory as the telecommunications revolution and de-industrialization in advanced economies has further segregated North America from the places of production for most consumer goods \citep{bryson1999economic}.    

This pivot of the American economy away from traditional manufacturing has led to the embracing the newer knowledge economy paradigm - centred around Andreessen's quip that "software is eating the world" \cite{Andressen2011}, lean manufacturing and faster equipment depreciation.  As a consequence, companies are also moving away from debt financed expansion to selling equity as the preferred method of raising capital. 	\cite{Graves2003} argues that this shift is accelerated by the new economy's lack of physical assets, such as tooling, inventory and real estate that can be used as loan collateral. This has an important effect with regards to the equity market and makes initial public offerings (IPOs) more important than ever. While investor to investor trades after the IPO does not raise capital for the firm in question, the price discovery mechanism of the stock market allows for sensible pricing of secondary offerings which do raise money for the firm \citep{Tobin1969}. The largest holders of these stocks are institutional investors.  

\section{A Thought Experiment}

As a thought experiment, what if we were to take an esteemed economic geographer from 1991 and transposed them to the year 2020, what would surprise them about the American financial system in the intervening 30 years?  One might suspect that this scene would resemble that of Micheal Douglas's character Gordon Gekko at the beginning of the movie WallStreet: Money Never Sleeps at which the disgraced former stockbroker Gordon Gekko leaving prison in which is antiquated 1980s personal effects are updated to the contemporary electronics.  After the culture shock of 30 years of technological advancement, how foreign would such a person find modern institutional investment?  

If this hypothetical person were to compare the evidence in this thesis to the world they knew in 1991, they would conclude that nothing and everything changed.  Nothing changed, in the sense that the top tiers of American metro areas by funds under management did not change much, New York is still the unquestioned occupant of the first tier, Boston and Chicago are still in the second tier and San Francisco and LA are in tier 2.5 today rather than tier 3.  On the other hand, everything changed, since the dollar amounts invested by institutional investors are at record highs and have more digits to the numbers - some holdings such as State Street have surpassed the trillion dollar mark.  

In the late 1980s, the trend was for firms to leave their metro area's central business district for the lower cost, lower tax and safer suburban office parks \citep{bodenmanfirm2000}.  A part of the urban decay and renewal can be attributed to the lead-crime hypothesis, in which leaded gasoline is responsible for a sizable amount of crime due to environmental lead poisoning via automobile exhaust \citep{feigenbaum2016lead,NBERw23392}.  The turnaround and revitalization of urban areas is also congruent with the 20 year lag over the reduction of urban pollution.  This urban renewal allows for a return to the original role of cities, that of being a nexus of trade and information. 

Similarly, from the point of view of this hypothetical economic geographer, the rise of the Asian economies might be breathtaking, even if many of the preconditions for this growth were established in the 1980s.  And yet, the most important Asian economy isn't Japan, as would have been expected in 1991, but the People's Republic of China. 












- Chapter II will examine the literature of institutional investment Lit review.  Both the geography literature, and the more recent finance and business literature.   

- Chapter III Data Pipeline

- Chapter IV Exploring the data 

- Chapter V Time Cube

- Chapter VI LDA plus Shift Share

Goals 