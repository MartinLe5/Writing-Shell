\chapter{Introduction}
\label{ChapterI}


Promts:


The aftermath of the Great Financial Crisis of 2008-2009 has shown that while most mainstream models in economics treated financial institutions as mere value-neutral plumbing - albeit plumbing that grew at rates much faster than the house in the words of Dr. Mark Blythe \todo{youtube talk citation}; it is ignored at their own peril.  This brings us to the research - specifically in geography - with regards to institutional investors. There hasn't been a lot of new literature in geography about the locational preferences of institutional investors during the 21rst century.  Part of this can be explained by the culture turn in Geography and its shifting the meta narrative of research away from quantitative surveys to more personalized explorations of space and place as well as the reduced perceived importance of classical location theory as the telecommunications revolution and de-industrialization in advanced economies has further segregated North America from the places of production for most consumer goods.    

This pivot of the American economy away from traditional manufacturing has led to the embracing the newer knowledge economy paradigm - centred around Andresen's quip that "software is eating the world" \cite{Andressen2011}, lean manufacturing and faster equipment depreciation.  As a consequence, companies are also moving away from debt financed expansion to selling equity as the preferred method of raising capital. 	\cite{Graves2003} argues that this shift is accelerated by the new economy's lack of physical assets, such as tooling, inventory and real estate that can be used as loan collateral. This has an important effect with regards to the equity market and makes initial public offerings (IPOs) more important than ever. While investor to investor trades after the IPO does not raise capital for the firm in question, the price discovery mechanism of the stock market allows for sensible pricing of secondary offerings which do raise money for the firm \citep{Tobin1969}.   The largest holders of these stocks are institutional investors.  

Institutional investors are individuals (corporate or meatbags) that exercise investment discretion over large sums of money.  





As such, with a few notable exceptions such as \cite{Graves2003,gongthe2012} and \cite{GreenOLef2014}, most of the Geography based literature for institutional investment dates to the 1980s and 1990s.  At the same time, the 2000s and the 2010s has seen some economics and finance articles that investigated the role of space and place with regards to investing, most of these papers research question is more interested in drawing a competitive advantage rather than survey the location of investors.  







- chapter II Lit review

- Chapter III Data Pipeline

- Chapter IV EDA

- Chapter V Time Cube

- Chapter VI LDA plus Shift Share

Goals 