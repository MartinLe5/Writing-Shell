	Commenting on the pivot of the American economy away from traditional manufacturing to embracing the newer knowledge economy paradigm, lean manufacturing and faster equipment depreciation.  As a consequence, companies are also moving away from debt financed expansion to selling equity as the preferred method of raising capital. 	\cite{Graves2003} argues that this shift is accelerated by the new economy's lack of physical assets, such as tooling, inventory and real estate that can be used as loan collateral. This has an important effect with regards to the equity market and makes initial public offerings (IPOs) more important than ever. While investor to investor trades after the IPO does not raise capital for the firm in question, the price discovery mechanism of the stock market allows for sensible pricing of secondary offerings which do raise money for the firm \citep{Tobin1969}.  
	
	
	The aftermath of the Great Financial Crisis of 2008-2009 has shown that assuming that lending institutions as mere value-neutral plumbing is an axiom that one adopts at their own peril \citep{Krugman2012blog}.  Especially when said plumbing is growing at a rate that is much faster than the house \todo{youtube talk citation}.  