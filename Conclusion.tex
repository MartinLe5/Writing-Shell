\chapter {Conclusion}

\subsection{Chapter II summary}

-explored the literature between place and investing, and the death of distance talking point
-examined the long shadow that efficient market hypothesis plays on investment research
\subsection{Chapter III summary}
-define 13F-HR
-define the data set that the following chapters will work with

\subsection{Chapter IIIb summary}
-Exploring the data
-find that investment firms are located mostly in urban locations, and are highly concentrated.
-Firms, once set up are mostly sedentary
- NYC might be loosing market share with regards to number of institutional investor firms in the USA, it is still the number one location in absolute terms, and it's not even close. 
However, there is a move in the wake of the great financial crisis towards more peripheral counties, but still in the main metro areas.  
Used the spherical application of Ripley's K-Function to measure the level of concentration and dispersion at various distance bands.  

Used the Gravity Model of Trade to explore inter-county investment patterns in the USA. Finds a distance decay in investments.  Furthermore, population size is a big predictor of investment flows.


\subsection{Chapter IV summary}

Space-time cube



\subsection{Chapter V summary}


idea for future research: Tying the 13-F H/R's political contributions to their portfolio, and see if there is a correlation between their political contributions and investment archetypes. (see coastal differences) 

Examine how the events of the past few years, such as examine if Brexit's influenced the British banking system can be seen in British based 13F-HR filers

- How the massive 5 trillion dollar money coccaine line provided to the Fed by congress in response to the COIVD-19 epidemic and the uncoupling of stock market performance from the real economy (tie back to , 