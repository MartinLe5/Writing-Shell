\chapter {Conclusion}
In the end, it would be fair for the esteemed geographer mentioned in Chapter \ref{ChapterI} to conclude that in the worlds of the French journalist and publisher Jean-Baptiste Alphonse Karr, \textit{plus \c{c}a change, plus c'est la m\^{e}me chose}.

The main goal of this paper was to update the literature on the locational preferences of institutional investors in the United States of America for a twenty-year long time period.  The existing literature, especially from a Geography perspective declines precipitously after the mid 1990s, reflecting the culture turn in geographic research.  The culture turn's emphasis on the human decision making processes and how humans interact with their environment coincided with a period of intellectual colonization by economists, who once again discovered the role of distance in their trade models \citep{scotta2004}.  This second source of research from economics and financial professionals is more up to date than the geography literature, but often elides over or omits important considerations for the geographer, and stands in the stifling shadow of the Efficient Market Hypothesis.  A third source for the decline of geographic research is tied to the so called "death of distance" hypothesis that is the hallmark of certain techno-futurists \citep{Obrian1992}.  Their belief is that the telecommunications revolution has effectively replaced "space" with "place", and that social networks are more important than proximity, falls somewhat short as practiced during this study period.  While the telecommunications revolution of the 1970s and 1980s unshackled financial operations from city centres that were then undergoing a wave of urban blight and dis-investment\footnote{The only increases in spending, when adjusted for inflation, that North-Eastern and Mid-West cities have seen is in the police budget \citep{derenoncourt2019can}}, the data examined here shows that the twenty-first century partially reversed that trend of suburbanization of financial institutions. 

While \cite{gongthe2012} showed that New York's FIRE sector was very resilient in the face of the 9/11 terrorist attack on the World Trade Centre in New York, there was still that underlying belief that the trend of suburbanization and fleeing to lower-tax jurisdiction would ultimately doom New York's financial sector.  The exploratory data analysis using graphing techniques proposed by  \cite{tufte1998visual}, as well as the spherical application of Ripley's K and the Gravity Model of Trade done in Chapter \ref{ChapterIIIb} confirms Gong and Keenan's finding of a resilient New York, for no other jurisdiction was remotely close in terms of adding new institutional investors in absolute numbers.  That being said, this same data does show that New York is loosing pace to a multitude of regional centres in relative number of investors.  If one were to ignore New York's continued edge in absolute terms of new institutional investors, this would nearly be a perfect example of  stage III of Quaternary Location Theory, where the regional headquarters are catching up to the national headquarter \citep{Semple_Phipps82}.  Finally, the Gravity Model of Trade is applied to inter-county investment flows data for the United States of America for the years 2013 to 2018.  The model's output shows that population count in the home and host county is a key factor in determining the amount of investment flows between these counties.  This stands in contrast to \cite{greena1993} and \cite{GreenOLef2014}, however this study's application of the Gravity Model of Trade included a much larger swath of the population, and thus showing the importance of the mid-tier cities and their "stadium-scale banks" as well as other financial firms in this size range. This poses the question as to why certain lower-tier large cities such as Miami have a comparatively small institutional investor presence for their population.  

A key tenet in Paul Krugmans's New Economic Geography was the role of increasing returns to scale \citep{krugman1991increasing}.  That is to say that early advantage snowballs into continued prosperity.  Similarly, \cite{davis2002bones} find compelling evidence that these early advantages need large shocks, such as but not limited to fundamental changes in underling patterns of trade, in order to disrupt the long-term growth of a sector.  For example, their paper finds that many of the key cities in Japan's economy today were mostly the same cities that were fundamental to Japan's economy during the Sengoku-Jidai civil war and the rise of the Bamboo Curtain in 1615, and that the massive disruptions cause by Curtis LeMay's aerial campaign during World War II did not significantly change the long-term economic growth of Japan.  Similarly, a point density map of US-based financial investors for the year 2018 would not bring that many surprises to somebody who was familiar with the location of institutional investment in the 1990s.  But within this overarching theme of continuity much change can lie hidden.  Therefore, in order to tease out patterns in the creation of new institutional investors, ESRI's time-cube analysis module can offer insights about emerging data by looking at space as well as time. 

The national hot-spot analysis highlights the location of State capitals in the flyover states which as mentioned previously are home to "stadium sized banks" as well as State employee pension funds.  That being said, while these veritable islands of high concentrations of funds under management among the vast American landscape were expected, they still paled in comparison to the major hot-spots of the BOS-NY-Wash metropolis, as well as the San Francisco Bay area and Southern California.   Digging deeper into the cities of Boston, Chicago, Los Angeles, New York and San Francisco, it is apparent that while these cities may have a multitude of suburban office parks in which institutional investment is carried out, the largest hot-spots of activities are in the Downtown core of these cities, and this remained true whether one looked at the longer scale phone book database, or the funds under management database.  This is consistent with the Ripley's k and gini coefficient analyses done in previous chapters, thus confirming that this advanced command and control function of the economy is of a decidedly urban nature.  
     
The last chapter takes a bit of a departure from the classical institutional investor literature.  Since there is broad homogenization of best practices among different types of institutional investors, from banks, hedge funds and insurance companies, this paper explored whether a classification scheme based on investor portfolio archetypes would provide novel insights into the geography of institutional investment.  The Latent Dirichlet allocation topic model of the holdings of institutional investors for the time period of 2013-2018 shows that there isn't a large pattern of regional specialization in investment strategies.  However, a shift-share analysis of the holdings when weighted by investment strategy reveals a sharp contrast between New York State and California.  New York showed a very commanding position at the centre of the American investment world, showing very high levels of returns when adjusting for the performance of their peers, and California on the other hand showed dismal performance.  This edges along a major weakness of the 13F-HR database, in that it only contains information on the holdings of publicly traded companies by institutional investors, and not their private equity or venture capital activities.  Viewing this data in light of California-based investors penchant for venture capital investing, this provides an easy explanation for California's poor performance (the money was outside the scope of view of the database). 














idea for future research: Tying the 13F-HR's political contributions to their portfolio, and see if there is a correlation between their political contributions and investment archetypes.

Examine how the events of the past few years, such as examine if Brexit's influenced the British banking system can be seen in British based 13F-HR filers

- How the massive 5 trillion dollar money coccaine line provided to the Fed by congress in response to the COIVD-19 epidemic and the uncoupling of stock market performance from the real economy (tie back to lack of good investments and money sloshing around). 










- \cite{harrison2019venture}[p.26] "[...]increasingly we see more and more sophisticated analytical sledgehammers being used to crack increasingly small and trivial research question nuts"

