\chapter {Conclusion}
In the end, it would be fair for the esteemed geographer mentioned in Chapter \ref{ChapterI} to conclude that in the words of the French journalist and publisher Jean-Baptiste Alphonse Karr, \textit{plus \c{c}a change, plus c'est la m\^{e}me chose} \citep[p.278]{Karr1864}.

In order to find possible paths moving forward after the field of Geography-based institutional investment was to read and synthesise the previous works.  The existing literature, especially from a Geography perspective declines precipitously after the mid 1990s, reflecting the culture turn in geographic research.  The culture turn's emphasis on the human decision making processes and how humans interact with their environment coincided with a period of intellectual colonization by economists, who once again discovered the role of distance in their trade models \citep{scotta2004}.  This second source of research from Economics, Business and Financial professionals is more up to date than the geography literature, but often elides over or omits important considerations for the geographer, and stands in the stifling shadow of the Efficient Market Hypothesis.  A third source for the decline of geographic research is tied to the so called "death of distance" hypothesis that is the hallmark of certain techno-futurists \citep{Obrian1992}.  Their belief is that the telecommunications revolution has effectively replaced "space" with "place", and that social networks are more important than proximity, falls somewhat short as practiced during this study period.  While the telecommunications revolution of the 1970s and 1980s unshackled financial operations from city centres that were then undergoing a wave of urban blight\footnote{ \cite{feigenbaum2016lead} and \cite{NBERw23392} suggest that wave of urban crime in the 1970s-90s is substantially explained by the presence of environmental pollution caused by the combustion of leaded gasoline.}  and dis-investment\footnote{The only increases in spending, when adjusted for inflation, that North-Eastern and Mid-West cities have seen is in the police budget \citep{derenoncourt2019can}.}, the data examined here shows that the 21st century partially reversed that trend of suburbanization of financial institutions.   


While \cite{gongthe2012} showed that New York's FIRE sector was very resilient in the face of the 9/11 terrorist attack on the World Trade Centre in New York, there was still that underlying belief that the trend of suburbanization and fleeing to lower-tax jurisdiction would ultimately doom New York's financial sector.  The exploratory data analysis using graphing techniques proposed by  \cite{tufte1998visual}, as well as the spherical application of Ripley's K and the Gravity Model of Trade done in Chapter \ref{ChapterIIIb} confirm Gong and Keenan's finding of a resilient New York, for no other jurisdiction was remotely close in terms of adding new institutional investors in absolute numbers. That being said, this same data does show that New York is losing pace to a multitude of regional centres in relative number of investors.  If one were to ignore New York's continued edge in absolute terms of new institutional investors, this would nearly be a perfect example of stage III of Quaternary Location Theory, where the regional headquarters are catching up to the national headquarter \citep{Semple_Phipps82}.  Finally, the Gravity Model of Trade is applied to inter-county investment flows data for the United States of America for the years 2013 to 2018.  The model's output shows that population count in the home and host county is a key factor in determining the amount of investment flows between these counties.  This stands in contrast to \cite{greena1993} and \cite{GreenOLef2014}, however this study's application of the Gravity Model of Trade included a much larger swath of the population, and thus showing the importance of the mid-tier cities and their "stadium-scale banks" as well as other financial firms in this size range. This poses the question as to why certain lower-tier large cities such as Miami have a comparatively small institutional investor presence for their population.  

A key tenet in Paul Krugmans's New Economic Geography was the role of increasing returns to scale \citep{krugman1991increasing}.  That is to say that early advantage snowballs into continued prosperity.  Similarly, \cite{davis2002bones} find compelling evidence that these early advantages need large shocks, such as but not limited to fundamental changes in underling patterns of trade, in order to disrupt the long-term growth of a sector.  For example, their paper finds that many of the key cities in Japan's economy today were mostly the same cities that were fundamental to Japan's economy during the Sengoku period (1467 -- 1615).  Massive disruptions, such as those cause by Curtis LeMay's aerial campaign during World War II, did not significantly change the long-term economic growth of Japan.  Similarly, a point density map of US-based financial investors for the year 2018 would not bring that many surprises to somebody who was familiar with the location of institutional investment in the 1990s.  This observation helps answer the question posed in \cite{GreenOLef2014} - whether the generative process of new investment firms will help cement or undermine the current spatial pattern of institutional investment.  On a macro scale, this question appears to be answered by the observations in Chapter \ref{ChapterIIIb} which demonstrate that new points mostly reflect the existing pattern, except for a relative decline of New York City, and an increase in the sunbelt as well as an increase in the number of internationally located institutional investors.  However, with regards to internationally based 13-F HR investors, \cite{Lefebvre2014} shows that foreign-based investors are drawn from a different distribution with regards to funds under management - the average foreign-based fund is larger than the average domestic institutional investor. One can see the logic,  considering the scale needed to compensate for the extra effort needed to collect and act on information that is by definition not in your home country \citep{malloythe2005}. That being said, within this overarching theme of continuity much change can lie hidden.  Therefore, in order to tease out patterns in the creation of new institutional investors, ESRI's time-cube analysis module can offer insights about emerging data by looking at space as well as time. 

The national hot-spot analysis highlights the location of State capitals in the flyover states which as mentioned previously are home to "stadium sized banks" as well as State employee pension funds.  That being said, while these veritable islands of high concentrations of funds under management among the vast American landscape were expected, they still paled in comparison to the major hot-spots of the BOS-NY-Wash metropolis, as well as the San Francisco Bay area and Southern California.   Digging deeper into the cities of Boston, Chicago, Los Angeles, New York and San Francisco, it is apparent that while these cities may have a multitude of suburban office parks in which institutional investment is carried out, the largest hot-spots of investors are located in the Downtown core of these cities, and this remained true whether one looked at the longer scale phone book database or the funds under management database.  This is consistent with the Ripley's K and Gini coefficient analyses done in previous chapters, thus confirming that this advanced command and control function of the economy is of a decidedly urban nature.  
     
The last chapter takes a bit of a departure from the classical institutional investor literature.  Since there is broad homogenization of best practices among different types of institutional investors, from banks, hedge funds and insurance companies, this paper explored whether a classification scheme based on investor portfolio archetypes would provide novel insights into the geography of institutional investment.  The Latent Dirichlet allocation Topic Model of the holdings of institutional investors for the time period of 2013-2018 shows that there isn't a large pattern of regional specialization in investment strategies.  However, a shift-share analysis of the holdings when weighted by investment strategy reveals a sharp contrast between New York State and California.  New York showed a very commanding position at the centre of the American investment world, showing very high levels of returns when adjusting for the performance of their peers, and California on the other hand showed dismal performance.  This edges along a major weakness of the 13F-HR database, in that it only contains information on the holdings of publicly traded companies by institutional investors, and not their private equity or venture capital activities.  Viewing this data in light of California-based investors penchant for venture capital investing, this provides an easy explanation for California's poor performance (the money was outside the scope of view of the database). 

The esteemed geographer then asks how will the reactions to the COVID-19 global pandemic affect investing?  The self-isolation bought about by the coronavirus has accelerated plans for employees working from home.  How much will this affect the benefits of co-location, especially in high rent spaces such as Manhattan?  Will the emptying out of Manhattan due to the self isolation's affect plans to choose Manhattan as a destination for new firms?   Time will tell, but looking at the past 50 years of location choices, and the theoretical frameworks developed by \cite{krugman1991increasing} and \cite{davis2002bones}, calls for Manhattan's decline may once again be premature.


